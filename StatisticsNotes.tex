\documentclass{article}

\usepackage{setspace}
\usepackage{listings}
\usepackage{appendix}
\usepackage{float}
\usepackage{amsfonts}
\usepackage{amsmath}
\usepackage[american]{babel}
\usepackage{pythontex}
\usepackage{graphicx}
\usepackage[hyperref=true, backref=true, backend=bibtex8]{biblatex}
\usepackage{csquotes}
\usepackage[pdftex, pdfusetitle, colorlinks, 
		urlcolor=blue, 
		filecolor=blue, 
		linkcolor=blue,
		citecolor=blue,]{hyperref}

\author{Nicholas Capo}
\date{\today}
\title{Statistics Class Notes}

\bibliography{textbook}

\begin{document}
\maketitle
\newpage
\tableofcontents
\newpage

\subsection{Definition of Statistics}

\enquote{Statistics is the science of collecting, organizing, analyzing, and interpreting data in order to make decisions.}

\section{Data}

\subsection{Data Sets}
\begin{description}
\item[Population] The collection of all outcomes, responses, measurements, or counts, that are of interest.

\item[Sample] A subset of the population.

\item[Parameter] A number that describes a population characteristic.

\item[Statistic] A number that describes a sample characteristic.
\end{description}

\subsection{Types of Data}

\begin{description}
\item[Qualitative Data] Attributes, labels, or non-numerical entries.

\item[Quantitative Data] Numerical measurements or counts.
\end{description}

\section{Sample Mean and Median}

\subsection{Definition}
\begin{description}
\item[Sample Mean] The average of the sample data points, however it may not be a data point.
$$\overline{x} = \sum_{i=1}^n\frac{x_i}{n} = \frac{x_1+x_2+x_3\cdots x_n}{n}$$
\item[Sample Median] The middle value of the data.

$$\tilde{x}=\left\{
\begin{matrix}
x_{(\frac{n+1}{2})} & \text{if $n$ is odd}\\
\frac{1}{2}(x_{\frac{n}{2}}+x_{\frac{n}{2}+1}) & \text{if $n$ is even}
\end{matrix}
\right.$$

\item[Trimmed Mean] A trimmed mean is computed by trimming off the largest and smallest set of values. For example a 10\% trimmed mean is found by eliminating the largest 10\% and smallest 10\% and computing the mean of the remaining values. This may be useful for data that contains possible outliers. Denoted by $x_{tr(\text{percent})}$
\end{description}

\section{Measures of Variability}

\subsection{Standard Deviation}

\subsubsection{Sample Variance}

$$s^2 = \sum_{i=1}^n \frac{(x_i - \overline{x})^2}{n-1}$$

\subsubsection{Sample Standard Deviation}

$$s=+\sqrt{s^2}$$

The standard deviation is $0$ when all the data points are the same.

\section{Descriptive Statistics}

\subsection{Quartiles}

Quartiles approximately divide an ordered data set into four equal parts.

\begin{description}
\item[First Quartile, $Q_1$]
About $25\%$ of the data fall on or below $Q_1$
\item[Second Quartile, $Q_2$]
About $50\%$ of the data fall on or below $Q_2$
\item[Third Quartile, $Q_3$]
About $75\%$ of the data fall on or below $Q_3$
\end{description}

\subsection{Range and Interquartile Range}

\subsubsection{Range}

$$\text{range} = \text{max value} - \text{min value}$$

\subsubsection{Interquartile Range}

$$IQR=Q_3 - Q_1$$

To help find outliers, compute $1.5 \times IQR$, and any values that lie outside the interval $[Q_1-1.5 \times IQR, Q_3+1.5 \times IQR]$ is a possible (and probable) outlier.

\subsection{Box and Whisker Plot}

Exploratory Data Analysis Tool

\begin{itemize}
\item Requires
	\begin{itemize}
	\item Min
	\item $Q_1$
	\item Median
	\item $Q_3$
	\item Max
	\end{itemize}
\end{itemize}

\begin{pycode}
import pylab
data = [1, 2, 3, 4, 5, 6, 11]
pylab.figure(figsize=(5,3))
pylab.boxplot(data, vert=0, sym='bx')
pylab.savefig('whiskerplot.pdf', bbox_inches='tight', orientation='landscape')
\end{pycode}

\subsubsection{Example}
Example Data: \pyc{print(data)}
\begin{center}
\includegraphics{whiskerplot}
\end{center}

\section{Homework}
\begin{itemize}
\item p. 13 \#'s 1.5, 1.6
\item p. 17 \#'s 1.11, 1.12
\item p. 31 \#'s 1.18, 1.19, 1.20, 1.29, 1.30
\end{itemize}


\newpage
\nocite{textbook}
\printbibliography

\end{document}
