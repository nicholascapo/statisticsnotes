\documentclass{book}

\usepackage{setspace}
\usepackage{listings}
\usepackage{appendix}
\usepackage{float}
\usepackage{amsfonts}
\usepackage{amsmath}
\usepackage[american]{babel}
\usepackage{pythontex}
\usepackage{graphicx}
\usepackage[hyperref=true, backref=true, backend=bibtex8]{biblatex}
\usepackage{csquotes}
\usepackage[pdftex, pdfusetitle, colorlinks, 
		urlcolor=blue, 
		filecolor=blue, 
		linkcolor=blue,
		citecolor=blue,]{hyperref}
\usepackage{datetime}
\settimeformat{ampmtime}

\bibliography{../textbook}

\title{\textsc{Class Notes\\ for \\ Elementary Statistics\\ Letu Math--1423}}
\author{Nicholas Capo \\ \href{mailto:nicholas.capo@gmail.com}{nicholas.capo@gmail.com}}
\date{\today\\ \currenttime}

\bibliography{textbook}

\begin{document}
\maketitle
\tableofcontents
\pagebreak

\section*{}
\begin{center}
This document comprises classroom notes from Statistics Class\\ at \href{letu.edu}{LeTourneau University}, in the Fall of 2012.

\vspace{10pt}

Although the author will attempt to be complete and correct in these notes, it is the readers responsibility to learn and understand the material. The author assumes no responsibility for the completeness or accuracy of this content. 

\vspace{10pt}

The latest version of this document is available at:\\ \url{https://bitbucket.org/nicholascapo/statisticsnotes/src/tip/StatisticsNotes.pdf}
\end{center}

\vfill
\begin{tabular}{c}
Copyright \copyright\ 2012 Nicholas Capo\\
Licensed under a {Creative Commons Attribution-ShareAlike 3.0 Unported}\\
\url{http://creativecommons.org/licenses/by-sa/3.0/}\\
\end{tabular}


\newpage
\chapter{Introduction}
\begin{center}
\textbf{Definition of Statistics}\\
\enquote{Statistics is the science of collecting, organizing, analyzing, and interpreting data in order to make decisions.}
\end{center}

\section{Data}

\subsection{Data Sets}
\begin{description}
\item[Population] The collection of all outcomes, responses, measurements, or counts, that are of interest.

\item[Sample] A subset of the population.

\item[Parameter] A number that describes a population characteristic.

\item[Statistic] A number that describes a sample characteristic.
\end{description}

\subsection{Types of Data}

\begin{description}
\item[Qualitative Data] Attributes, labels, or non-numerical entries.

\item[Quantitative Data] Numerical measurements or counts.
\end{description}

\section{Sample Mean and Median}

\subsection{Definition}
\begin{description}
\item[Sample Mean] The average of the sample data points, however it may not be a data point.
$$\overline{x} = \sum_{i=1}^n\frac{x_i}{n} = \frac{x_1+x_2+x_3\cdots x_n}{n}$$
\item[Sample Median] The middle value of the data.

$$\tilde{x}=\left\{
\begin{matrix}
x_{(\frac{n+1}{2})} & \text{if $n$ is odd}\\
\frac{1}{2}(x_{\frac{n}{2}}+x_{\frac{n}{2}+1}) & \text{if $n$ is even}
\end{matrix}
\right.$$

\item[Trimmed Mean] A trimmed mean is computed by trimming off the largest and smallest set of values. For example a 10\% trimmed mean is found by eliminating the largest 10\% and smallest 10\% and computing the mean of the remaining values. This may be useful for data that contains possible outliers. Denoted by $x_{tr(\text{percent})}$
\end{description}

\section{Measures of Variability}

\subsection{Standard Deviation}

\subsubsection{Sample Variance}

$$s^2 = \sum_{i=1}^n \frac{(x_i - \overline{x})^2}{n-1}$$

\subsubsection{Sample Standard Deviation}

$$s=+\sqrt{s^2}$$

The standard deviation is $0$ when all the data points are the same.

\section{Descriptive Statistics}

\subsection{Quartiles}

Quartiles approximately divide an ordered data set into four equal parts.

\begin{description}
\item[First Quartile, $Q_1$]
About $25\%$ of the data fall on or below $Q_1$
\item[Second Quartile, $Q_2$]
About $50\%$ of the data fall on or below $Q_2$
\item[Third Quartile, $Q_3$]
About $75\%$ of the data fall on or below $Q_3$
\end{description}

\subsection{Range and Interquartile Range}

\subsubsection{Range}

$$\text{range} = \text{max value} - \text{min value}$$

\subsubsection{Interquartile Range}

$$IQR=Q_3 - Q_1$$

To help find outliers, compute $1.5 \times IQR$, and any values that lie outside the interval $[Q_1-1.5 \times IQR, Q_3+1.5 \times IQR]$ is a possible (and probable) outlier.

\subsection{Box and Whisker Plot}

Exploratory Data Analysis Tool

\begin{itemize}
\item Requires
	\begin{itemize}
	\item Min
	\item $Q_1$
	\item Median
	\item $Q_3$
	\item Max
	\end{itemize}
\end{itemize}

\begin{pycode}
import pylab
data = [1, 2, 3, 4, 5, 6, 11]
pylab.figure(figsize=(5,2))
pylab.boxplot(data, vert=0, sym='bx')
pylab.savefig('whiskerplot.pdf', bbox_inches='tight', orientation='landscape')
sdata = sorted(data)
min = sdata[0]
outlier = sdata[-1]
max = sdata[-2]
median = pylab.median(sdata)

\end{pycode}

\subsubsection{Example}

\begin{tabular}{ll}
Example Data &\py{data}\\
Min &\py{min}\\
Median & \py{median}\\
Max & \py{max}\\
Outlier & \py{outlier}\\
\end{tabular}


\begin{figure}[H]
\begin{center}
\includegraphics[width=.75\textwidth]{whiskerplot}
\end{center}
\caption{Example Box And Whisker Plot}
\end{figure}

\section{Stem and Leaf Plots}

These look like a sideways histogram

Data: \py{[31, 21, 32, 33, 41, 42, 58, 25, 21]}\\

\begin{tabular}{r|ll}
Stem & Leaf & Key: $a|b=ab$\\
\hline
2&1,1,5\\
3&1,2,3\\
4&1,2\\
5&8\\
\end{tabular}

\subsection{Key Notation}

Key: 4|5 = 45
Key: 4|5 = 4.5

\subsection{Double Stem and Leaf}

Separate the leaves into two groups, (0-4, and 5-9)

Data: \py{[31, 21, 32, 33, 41, 42, 58, 25, 21]}\\

\begin{tabular}{r|ll}
Stem & Leaf & Key: $a|b=ab$\\
\hline
2&1,1\\
2&5\\
3&1,2,3\\
4&1.2\\
4&\\
5&\\
5&8\\
\end{tabular}

\section{Frequency Distribution}
A table that shows classes or intervals of data with a count of the number of entries in each class.

\subsection{Midpoint of a Class}
Average of the class limits.
$$\frac{(\text{lower class limit})+(\text{upper class limit})}{2}$$

\subsection{Relative Frequency}
$$\frac{\text{class frequency}}{\text{sample size}}=\frac{f}{n}$$

\section{Scatter Plots}
Each entry in one data set corresponds to one entry in a second set, one-to-one mapping.

\subsection{Example Scatter Plot}

\begin{pycode}
import pylab
import random
count = 12
data = []
for i in range(count): data.append(random.randint(1, count))
x = range(1, count+1)

pylab.figure(figsize=(5, 5))

pylab.scatter(x, data, label='(X,Y)')

pylab.legend()
pylab.savefig('scatter.pdf', orientation='landscape')
\end{pycode}

\begin{tabular}{ll}
Data:&\\
X: & \texttt{\py{x}}\\
Y: & \texttt{\py{data}}\\
\end{tabular}

\begin{figure}[H]
\begin{center}
\includegraphics[width=.75\textwidth]{scatter}
\end{center}
\caption{Example Scatter Plot}
\end{figure}

\section{Homework}
\begin{itemize}
\item Page 13 \#'s 1.5, 1.6
\item Page 17 \#'s 1.11, 1.12
\item Page 31 \#'s 1.18, 1.19, 1.20, 1.29, 1.30
\end{itemize}

%%%%%%%%%%%%%%%%%%%%%%%%%%%%%%%%%%
\chapter{Probability}

\section{Experiments}
Any process that generates a set of data.

\section{Sample Space}
The set of all possible outcomes of a statistical experiment, denoted $S$. The sample space with no elements is the empty set or null set, denoted $\emptyset$

\subsection{Example}

$$S = \{ 3, 2, 1, 0\}$$
$$S = \{ x | 0 < x < 25 \}$$
$$S= \{ x^2 | x \in \mathbb{R}\}$$

\subsection{Tree Diagrams}
A Tree Diagram can be used to list all possible outcomes

\subsection{Events}
An event is a subset of a sample space. The null set ($\emptyset$) and the sample space ($S$) are both subsets of the sample space $S$.

\subsubsection{Intersection}
The intersection of two events $A$ and $B$, denoted $A \cap B$, is the event containing all elements that are common to $A$ and $B$. If $A \cap B = \emptyset$ than $A$ and $B$ are called mutually exclusive or disjoint.

\subsubsection{Union}
The union of two events $A$ and $B$, denoted $A \cup B$, is the event containing all elements that belong to $A$ or $B$ or both.

\subsubsection{Compliment}
The compliment of an event $A$ with respect to $S$ is a subset of all elements of $S$ not in $A$, denoted $A'$

\section{Homework}
\begin{itemize}
\item Page 42 \#'s 2.3, 2.6, 2.10, 2.11, 2.14, 2.16, 2.18
\end{itemize}




%%%%%%%%%%%%%%%%%%%%%%%%%%%%%%%%%%
\newpage
\nocite{textbook}
\printbibliography

\end{document}
