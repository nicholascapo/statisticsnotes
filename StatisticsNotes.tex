\documentclass{article}

\usepackage{setspace}
\usepackage{listings}
\usepackage{appendix}
\usepackage{float}
\usepackage{amsfonts}
\usepackage{amsmath}
\usepackage[american]{babel}
\usepackage[hyperref=true, backref=true, backend=bibtex8]{biblatex}
\usepackage{csquotes}
\usepackage[pdftex, pdfusetitle, colorlinks, 
		urlcolor=blue, 
		filecolor=blue, 
		linkcolor=blue,
		citecolor=blue,]{hyperref}

\author{Nicholas Capo}
\date{\today}
\title{Statistics Class Notes}

\bibliography{textbook}

\begin{document}
\maketitle
\newpage
\tableofcontents
\newpage

\subsection{Definition of Statistics}

\enquote{Statistics is the science of collecting, organizing, analyzing, and interpreting data in order to make decisions.}

\section{Data}

\subsection{Data Sets}
\begin{description}
\item[Population] The collection of all outcomes, responses, measurements, or counts, that are of interest.

\item[Sample] A subset of the population.

\item[Parameter] A number that describes a population characteristic.

\item[Statistic] A number that describes a sample characteristic.
\end{description}

\subsection{Types of Data}

\begin{description}
\item[Qualitative Data] Attributes, labels, or non-numerical entries.

\item[Quantitative Data] Numerical measurements or counts.
\end{description}

\section{Sample Mean and Median}

\subsection{Definition}
\begin{description}
\item[Sample Mean] The average of the sample data points, however it may not be a data point.
$$\overline{x} = \sum_{i=1}^n\frac{x_i}{n} = \frac{x_1+x_2+x_3\cdots x_n}{n}$$
\item[Sample Median] The middle value of the data.

$$\tilde{x}=\left\{
\begin{matrix}
x+\frac{(x_{n+1})}{2} & \text{if n is odd}\\
\frac{1}{2}(x_{n-2}+x_{\frac{n}{2}+1}) & \text{if n is even}
\end{matrix}
\right.$$

\item[Trimmed Mean] A trimmed mean is computed by trimming off the largest and smallest set of values. For example a 10\% trimmed mean is found by eliminating the largest 10\% and smallest 10\% and computing the mean of the remaining values. This may be useful for data that contains possible outliers. Denoted by $x_{tr(\text{percent})}$
\end{description}

\section{Homework}
p. 13 \#'s 1.5, 1.6
p.17 \#'s 1.11, 1.12

\newpage
\nocite{textbook}
\printbibliography

\end{document}
